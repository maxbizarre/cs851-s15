\label{reference}
%%%%%%%%%%%%%%%%%%%%%%%% referenc.tex %%%%%%%%%%%%%%%%%%%%%%%%%%%%%%
% sample references
% "computer science"
%
% Use this file as a template for your own input.
%
%%%%%%%%%%%%%%%%%%%%%%%% Springer-Verlag %%%%%%%%%%%%%%%%%%%%%%%%%%

%
% BibTeX users please use
% \bibliographystyle{}
% \bibliography{}
%
% Non-BibTeX users please use
\begin{thebibliography}{99.}
%
% and use \bibitem to create references.
%
% Use the following syntax and markup for your references
%
% Monographs
%\bibitem{monograph} Kajan E (2002)
%Information technology encyclopedia and acronyms. Springer, Berlin
%Heidelberg New York

% Contributed Works
%\bibitem{contribution} Broy M (2002) Software engineering -- From
%auxiliary to key technologies. In: Broy M, Denert E (eds)
%Software Pioneers. Springer, Berlin Heidelberg New York

% Journal
%\bibitem{journal} Che M, Grellmann W, Seidler S (1997)
%Appl Polym Sci 64:1079--1090

% Theses
%\bibitem{thesis} Ross DW (1977) Lysosomes and storage diseases. MA
%Thesis, Columbia University, New York

\bibitem \hyperref[io]{https://docs.python.org/2/tutorial/inputoutput.html}

\bibitem \hyperref[for]{https://wiki.python.org/moin/ForLoop}

\bibitem \hyperref[if]{https://docs.python.org/2/tutorial/controlflow.html}

\bibitem \hyperref[ds]{https://docs.python.org/2/tutorial/datastructures.html}

\bibitem \hyperref[array]{http://stackoverflow.com/questions/1514553/how-to-declare-an-array-in-python}

\bibitem \hyperref[json]{https://docs.python.org/2/library/json.html}

\bibitem \hyperref[mean]{http://www.r-tutor.com/elementary-statistics/numerical-measures/mean}

\bibitem \hyperref[median]{https://stat.ethz.ch/R-manual/R-patched/library/stats/html/median.html}

\bibitem \hyperref[sd]{https://stat.ethz.ch/R-manual/R-patched/library/stats/html/sd.html}

\bibitem \hyperref[se]{http://stackoverflow.com/questions/2676554/in-r-how-to-find-the-standard-error-of-the-mean} 

\end{thebibliography}