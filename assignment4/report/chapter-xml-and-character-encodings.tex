\chapter{Violation of Good Practice: XML and character encodings}
\label{xml-and-character-encodings} 

The W3C does not consider it good practice to indicate the character encoding of XML documents in protocol headers because the data is self-describing.\cite{ArchOfWWW}

XML documents provide for self-description by specifying character set information, like so:
{\small
\begin{Verbatim}[commandchars=\\\{\}]
    <?xml version="1.0" \colorbox{yellow}{encoding="UTF-8"}?>
\end{Verbatim}
}

This is further complicated by the fact that the default encoding for documents with a content type starting with \emph{text/} is assumed to be US-ASCII unless otherwise specified,\cite{ArchOfWWW} so to help the web agent preserve the correct encoding, the \emph{charset} header \textbf{must} be given.

The concern is that the user agent is supposed to determine the character set based on content of the XML document, \textbf{not} from the metadata specified by \emph{Content-Type}.  What if the XML document specifies one encoding and the header specifies something else?  Which should the user agent obey?

CNN does not want to take any chances with the representation referenced by URI {\color{blue} http://rss.cnn.com/rss/cnn\_topstories.rss}, and specifies the encoding \emph{ISO-8859-1} in the headers anyway:
{\small
\begin{Verbatim}[commandchars=\\\{\}]
    22:45:19 smj@coyote:~/Documents/Projects/Current/CS751/Assignment1
    $ --> nc rss.cnn.com 80
    HEAD /rss/cnn_topstories.rss HTTP/1.1
    Connection: close
    Host: rss.cnn.com
    
    HTTP/1.1 200 OK
    Content-Type: text/xml; \colorbox{yellow}{charset=ISO-8859-1}
    ETag: 6f/V+d5nBJ7LtJ/yjXR1Pbt+IGk
    Last-Modified: Tue, 01 Feb 2011 03:46:02 GMT
    Date: Tue, 01 Feb 2011 03:46:30 GMT
    Expires: Tue, 01 Feb 2011 03:46:30 GMT
    Cache-Control: private, max-age=0
    X-Content-Type-Options: nosniff
    X-XSS-Protection: 1; mode=block
    Server: GSE
    Connection: close
\end{Verbatim}
}

The top of the document retrieved specifies its character set as well:
{\small
\begin{Verbatim}[commandchars=\\\{\}]
    <?xml version="1.0" encoding="ISO-8859-1"?>
\end{Verbatim}
}

Again, which is the web agent supposed to use, the header's \emph{charset} or the XML document's \emph{encoding}?  What if they turned out to be different?

This is only a problem with representations of XML documents provided with a content type of \emph{text/} specified in their protocol headers, as those using \emph{application/} do not need to specify encodings, nor should they because the XML document already contains the encoding.